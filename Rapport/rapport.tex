\documentclass{article}
\usepackage[francais]{babel}
%\usepackage[latin1]{inputenc}
\usepackage[utf8]{inputenc}  
%\usepackage[T1]{fontenc}
\usepackage[top = 3cm, left = 4cm, right = 4cm ]{geometry}
%\usepackage[pdftex]{graphicx}
\usepackage{fancyhdr}
%\usepackage{lscape}
%\usepackage[absolute]{textpos}
%\usepackage{amssymb}

\title{
{\Huge \textbf{Rapport de projet}\\
Algorithmique et bioinformatique}}

\author{\textbf{Sneessens Joachim}\\\textbf{Tulippe-Hecq Arnaud}\\MA1 Info}


\date{Année Académique 2019-2020\\
Master 1 en sciences informatiques\\
\vspace{1cm}
Faculté des Sciences, Université de Mons}


\pagestyle{fancy}
\lhead{Sneessens J. - Tulippe-Hecq A.}
\rhead{MA1 Info}
\cfoot{\thepage}

\begin{document}

\maketitle

\newpage

\section{Introduction}

\subsection{Enoncé du problème}

Ce projet consiste en la résolution d'un problème complexe lié à l'assemblage de fragments d'ADN.

\subsection{Rappel des objectifs}

Pour atteindre la résolution du problème, il faut concevoir un programme capable  à partir d'une collection de fragments de fournir une séquence cible. Ce programme doit :
\vspace{0.5cm}
\begin{itemize}
\item appliquer bonnes notions d'algorithmique et de bonnes pratique de programmation.
\item être optimisé en temps et en mémoire face à la grande quantité de donnée
\item utiliser l'appoximation de type Greedy pour l'assemblage et l'alignement semi-global pour la ressemblance entre fragments.
\item pourvoir être évalué sur ces résultats via l'outil de $dotmatcher$
\end{itemize} 

\section{Répartition du travail}

Voici une table résumant la répartition des étapes du projet.

\begin{table}[!htbp]
\begin{center}
\begin{tabular}{|p{3cm}||p{2.5cm}|p{3cm}|}
\hline
\textbf{Etape} & Catégorie & Gestion\\
\hline\hline
blabla & balbalbla & ablaba\\
\hline
\\
\hline
\\
\hline
\\
\hline
\\
\hline
\end{tabular}
\end{center}
\end{table}

\section{Etapes clés}

\subsection{OverlapGraph}

\subsection{Chemin hamiltonien}

\subsection{Propagation des gaps}

Pour classer les fragments, d'abord pensé aux abr équilibrés, ensuite tri par tas et puis hashmap

\subsection{Vote de consensus}

\section{Qualité du projet}

\subsection{Points faibles}

\subsection{Points forts}

\section{Conclusion}


\end{document}