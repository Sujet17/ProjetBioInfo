\documentclass{article}
\usepackage[french]{babel}
%\usepackage[latin1]{inputenc}
\usepackage[utf8]{inputenc}  
%\usepackage[T1]{fontenc}
\usepackage[top = 3cm, left = 4cm, right = 4cm ]{geometry}
\usepackage[pdftex]{graphicx}
\usepackage{fancyhdr}
\usepackage{colortbl}
\usepackage{xcolor}

\usepackage{moreverb}

\definecolor{lightgreen}{rgb}{.5,1,.5}
\definecolor{lightred}{rgb}{1,.5,.5}

\title{
{\Huge \textbf{Rapport de projet}\\
Algorithmique et bioinformatique}}

\author{\textbf{Sneessens Joachim}\\\textbf{Tulippe-Hecq Arnaud}\\MA1 Info}


\date{Année Académique 2019-2020\\
Master 1 en sciences informatiques\\
\vspace{1cm}
Faculté des Sciences, Université de Mons}


\pagestyle{fancy}
\lhead{Sneessens J. - Tulippe-Hecq A.}
\rhead{MA1 Info}
\cfoot{\thepage}

\begin{document}

\maketitle

\vspace{9cm}

\begin{center}
\begin{tabular}[t]{c c c}
\includegraphics[height=1.5cm]{logoumons.jpg} &
\hspace{1cm} &
\includegraphics[height=1.5cm]{logofs.jpg}
\end{tabular}
\end{center}



\newpage

\section{Introduction}

%Ce projet est réalisé dans le cadre du cours d'\textit{Algorithmique et BioInformatique} donné par le Prof. Olivier Delgrange.
%donné par les enseignants Olivier Delgrange et Clément Tamines, 

L'objectif de ce projet est l'implémentation d'une solution au problème d'assemblage de fragments d'ADN. Suite au séquençage de l'ADN, un ensemble de courts fragments est obtenu, et l'assemblage consiste à aligner/fusionner ces derniers afin de retrouver la séquence d'origine. Les fragments issus du séquençage peuvent provenir du brin $3' \to 5'$, ou de son complémentaire $5' \to 3'$, ainsi lorsqu'un fragment est présent dans l'ensemble issu du séquençage, son complémentaire inversé doit aussi être considéré dans la reconstruction de la séquence originale.

Plus précisément, le but de ce projet est de fournir un programme java prenant en entrée une collection de fragments et capable de reconstituer au mieux la séquence d'origine. Au vu de la grande taille des problèmes, une grande attention doit être portée à la complexité en temps et en mémoire. Certaines collections de fragments et leurs séquences cibles respectives sont à disposition pour permettre d'évaluer la qualité du programme (la comparaison entre la séquence cible et la séquence obtenue par le 
programme étant effectuée à l'aide de l'outil $dotmatcher$).
Précisons que l'ADN n'est vu et traité que sous son abstraction en temps que chaîne de caractères.

Les fragments reçus en entrée sont des chaînes des caractères (dont la longueur varie entre 500 et 700 caractères). Les différents caractères sont des $a$, $c$, $t$ et $g$ correspondant respectivement à adénine, cytosine, thymine et guanine, qui sont les différentes molécules de base constituant l'ADN.


\section{Organisation du travail}

Les différentes étapes du projet ont été implémentées dans l'ordre dans lequel elles sont présentées dans la section suivante. 
L'implémentation d'une étape ne commençait qu'une fois l'étape précédente terminée et testée. Plutôt que d'assigner à chaque membre du groupe des parties entières du travail, nous travaillions simultanément sur les mêmes parties du projet. En conséquence, il est impossible de mettre en évidence une répartition claire des tâches. Cette approche nous a permis de nous concentrer successivement sur chacune des parties du projet. S'il fallait changer quelque chose à une étape précédente, les tests unitaires nous permettaient de vérifier que les changements ne corrompent pas le programme. \textcolor{red}{ De plus, avoir tous deux connaissance de l'implémentation précise des autres parties du projet permet de m}

L'outil $git$ a été utilisé pour le contrôle de version, et le projet a été implémenté sous \textit{java 11}.


\newpage

\section{\'Etapes clés}

Les différentes étapes à suivre du projet nous ont été présentées lors de la séance de présentation du projet. Le prochain paragraphe décrit brièvement leurs places respectives dans l'assemblage de fragments.

La première étape de cet algorithme consiste à construire un graphe dit \textit{overlap graph} dont les sommets sont les fragments   et les arcs sont les poids des alignements semi-globaux entre ces fragments.
Ensuite vient la construction d'un chemin hamiltonien sur ce graphe par un algorithme glouton.
L'avant dernière étape consiste à aligner tous les fragments selon l'ordre obtenu dans le chemin hamiltonien en ajoutant des gaps si nécessaire. On parle de propagation des gaps.
Finalement intervient le vote de consensus qui, d'après l'ensemble de fragments alignés obtenus après la propagation des gaps, déterminera la séquence finale que retournera le programme.

Les sections suivantes sont consacrées à une explication plus approfondie de chacune de ces étapes clés, chacune décrivant d'abord précisément le rôle de l'étape avant de s'attarder sur les choix d'implémentation propres à notre projet.

\subsection{Alignement semi-global}

L'alignement semi-global entre deux fragments est utilisé pour construire l'\textit{overlap graph}. \textcolor{red}{L'alignement semi-global vise à mesurer la ressemblance entre deux fragments sans pénaliser le décalage.} 

C'est aussi à cette étape que sont détectées les inclusions entre fragments.

\textcolor{red}{
L'arc $f \to g$ dans l'\textit{overlap graph} aura pour poids le score de l'alignement semi-global dans lequel $f$ finit par des gaps et $g$ commence par des gaps lorsqu'ils sont alignés.}

L'alignement semi-global optimal est calculé par programmation dynamique. 
Le score de l'alignement entre deux fragments $f$ et $g$ est défini d'après les scores de l'alignement entre $f*$ et $g*$, où ces derniers sont les préfixes de taille $n-1$ respectifs de $f$ et $g$, i.e., $f*$ est $f$ sans son dernier caractère et similairement pour $g*$ et $g$.
La relation est la suivante:
$$score(f, g) = 
 \left\{
    \begin{array}{ll}
        score(f,g*)+gapPenalty \\
        score(f*,g*)+matchScore(f_{last}, g_{last})\\
        score(f*,g)+gapPenalty
    \end{array}
\right.$$
où 
\begin{itemize}
\item $gapPenalty$ est la pénalité de score liée à l'insertion d'un gap. Pour ce projet, cette pénalité vaut -2.
\item $f_{last}$ et $g_{last}$ sont les derniers caractères de $f$ et $g$.
\item $matchScore$ une fonction qui retournera le score d'un match si les deux caractères donnés en entrée sont identiques, et le score d'un mismatch sinon. Ces paramètres valent ici respectivement 1 et -1.
\end{itemize} 

\newpage
Le tableau suivant montre la matrice calculée pour déterminer les alignements $f \to g$ et $g \to f$, avec $f =$ acaatgatc et $g =$ caagatcagga (choisis arbitrairement).

\begin{table}[h]
	\centering
	\begin{tabular}{|c|c|c|c|c|c|c|c|c|c|c|c|c|c|}
		\hline
		&& C & A & A & G & A & T & C & A & G & G & A\\		
		\hline
		& 0 & 0 & 0 & 0 & 0 & 0 & 0 & 0 & 0 & 0 & \cellcolor{lightgreen}0 & 0  \\
		\hline 
		A&\cellcolor{lightred}0 & -1 & 1 & 1 & -1 & 1 & -1 & -1 & 1 & -1 & -1 &\cellcolor{lightgreen}1 \\
		\hline 
		C&0 & \cellcolor{lightred}1 & -1 & 0 & 0 & -1 & 0 & 0 & -1 & 0 & -2 & -1 \\
		\hline 
		A&0 & -1 & \cellcolor{lightred}2 & 0 & -1 & 1 & -1 & -1 & 1 & -1 & -1 & -1  \\
		\hline 
		A&0 & -1 & 0 & \cellcolor{lightred}3 & 1 & 0 & 0 & -2 & 0 & 0 & -2 & 0 \\
		\hline 
		T&0 & -1 & -2 & \cellcolor{lightred}1 & 2 & 0 & 1 & -1 & -2 & -1 & -1 & -2  \\
		\hline 
		G&0 & -1 & -2 & -1 & \cellcolor{lightred}2 & 1 & -1 & 0 & -2 & -1 & 0 & -2  \\
		\hline 
		A&0 & -1 & 0 & -1 & 0 & \cellcolor{lightred}3 & 1 & -1 & 1 & -1 & -2 & 1  \\
		\hline 
		T&0 & -1 & -2 & -1 & -2 & 1 & \cellcolor{lightred}4 & 2 & 0 & 0 & -2 & -1  \\
		\hline 
		C&0 & 1 & -1 & -3 & -2 & -1 & 2 & \cellcolor{lightred}5 & 3 & 1 & -1 & -3  \\
		\hline 
	\end{tabular}	
	\caption{\textcolor{red}{Table de calcul de l'alignement semi-global avec $f$ = acaatgatc et $g$ = caagatcagga}}
\end{table}

\textcolor{red}{L'alignement est reconstruit de la droite vers la gauche.}

La reconstitution de l'alignement $f \to g$ est mise en évidence en rouge dans le tableau. Pour trouver l'alignement $f \to g$, on  commence depuis la case de score maximal dans la dernière ligne. 
Partant de cette case, on va remonter le chemin vers la case $(0,0)$. Il s'agit de retrouver itérativement la case dont on vient parmi la case située à gauche, la case située au dessus et la case supérieure gauche.

\textcolor{blue}{
- Parler de comment on trouve les inclusions\\
- Dire que les fragments qu'on utilise sont des tableaux de byte\\
- La matrice de score est une matrice de short\\
- parler de la complexité en temps et en espace de cette étape
}

$f \to g$
\begin{verbatim}
acaatgatc----
-caa-gatcagga
\end{verbatim}
En partant de la i\up{e} case de la dernière ligne, on sait qu'on va aligner le dernier caractère de $f$ avec le i\up{e} caractère de $g$. 

L'alignement $g \to f$ est construit en suivant les mêmes règles, mais en partant de la case de score maximal de la dernière colonne. Il est mis en évidence en vert dans le tableau.
$g \to f$
\begin{verbatim}
caagatcagga--------
----------acaatgatc
\end{verbatim}


Cette étape nécessite de calculer et stocker $n*m$ scores, avec $n$ et $m$ les tailles de $f$ et $g$. Pour éviter l'utilisation inutile de mémoire, les scores dans la matrice sont \textit{short}.

\subsection{Overlap graph}

L'\textit{overlap graph} est un graphe orienté et pondéré dont les sommets sont des fragments. Les arcs sont déterminés d'après les alignements semi-globaux. Soit $w$ le score de l'alignement semi-global entre $f$ et $g$, alors l'arc du graphe allant de $f$ à $g$ aura un poids de $w$. Dans notre cas, les fragments données en entrée et leurs complémentaires-inversés sont les sommets du graphe.

Pour construire ce graphe, il faut rechercher l'alignement semi-global optimal pour chaque paire de fragments du graphe (excepté les paires constituées d'un fragment et son complémentaire-inversé).

De chaque sommet part un arc vers tous les autres - excepté son complémentaire-inversé. Il y a donc $n*(n-2)$ arcs à calculer (où $n$ est le nombre de sommets du graphe). On a pu voir à la section précédente qu'une seule matrice d'alignement permet de trouver à la fois l'arc $f\to g$ et l'arc $g \to f$. On sait aussi que les arcs $f \to g$ et $g' \to f'$  (avec $f'$ et $g'$ les complémentaires-inversés de $f$ et $g$) ont le même poids. Par conséquent, on peut calculer les poids des 8 arcs entre les sommets $f$, $f'$, $g$ et $g'$ en construisant uniquement deux matrices d'alignement semi-global. Au final $\frac{n*(n-2)}{8}$ matrices devront être construites.

\textcolor{red}{
S'il faut aussi considérer les fragments complémentaires-inversés dans le graphe, c'est parce les fragments issus du séquençage peuvent provenir du brin $3' \to 5'$ ou du brin $5' \to 3'$. Le problème d'assemblage de fragments consiste à reconstituer un seul des deux brins, ce qui demande d'écarter les fragments provenant du mauvais brin. }

Certains fragments du graphe peuvent être inclus à d'autres. Ceci s'explique par le fait que les fragments peuvent provenir du brin $3' \to 5'$ ou du brin $5' \to 3'$. Prenons le cas de deux fragments $f$ et $g$ provenant chacun d'un brin différent, et correspondant tous deux à la même zone de la double-hélice d'ADN. Alors $f$ sera très similaire à $g'$. Dans ce cas, l'inclusion sera détectée lors de l'alignement semi-global, et le fragment le plus court sera considéré comme inclus à l'autre. 
Nous avons pris le parti d'écarter du graphe un fragment dès que son inclusion à un autre fragment est détectée - le complémentaire-inversé est lui aussi écarté. C'est une manière non exacte d'écarter les fragments provenant du mauvais brin.

Le fait d'écarter les fragments inclus à d'autres a un grand impact sur le temps de calcul du graphe. En effet, pour chacune des collections de fragments, 80\% des fragments s'avéraient être inclus à d'autre. Cela réduit d'un facteur 5 le nombre de sommets du graphe et donc d'un facteur 25 le nombre d'arcs à calculer. En réalité, le gain n'est pas aussi important puisque les inclusions ne sont pas connues à l'avance, ce qui fait que certains arcs seront construits vers et depuis des fragments inclus. Néanmoins, une fois que l'inclusion a été détectée, plus aucun arc ne sera construit vers (ou depuis) le fragment concerné. Quant aux arcs construits inutilement, ils seront simplement ignorés.

Puisque le graphe est construit dans l'unique but de parcourir les arcs par ordre décroissant pour construire le chemin hamiltonien, nous avons choisi de mettre tous les arcs dans un tas (\textit{PriorityBlockingQueue}\footnote{https://docs.oracle.com/javase/7/docs/api/java/util/concurrent/PriorityBlockingQueue.html} de java) où ils sont triés en fonction de leur poids. 

Cette étape étant de loin la plus coûteuse en temps de l'assemblage de fragments, nous avons décidé de paralléliser les calculs des arcs pour exploiter au maximum les multiples cœurs du processeur. Ce multithreading est concrètement réalisé à l'aide des streams java (plus précisément les \textit{IntStream}\footnote{https://docs.oracle.com/javase/8/docs/api/java/util/stream/IntStream.html} qui gèrent eux-mêmes la parallélisation. Ce multithreading a imposé l'utilisation de structures de données \textit{threadsafe}, ce qui explique l'utilisation d'une \textit{PriorityBlockingQueue} pour stocker les arcs et d'un \textit{Vector} pour marquer les fragments inclus.

Dans le graphe tel que nous l'avons implémenté, chaque nœud d'un graphe contient un fragment $f$ et son complémentaire-inversé $f'$, plutôt que d'avoir deux nœuds distincts. Ce choix ne change rien à la logique du graphe. Cette implémentation est d'abord motivée par le fait qu'aucun arc n'est construit entre un fragment et son complémentaire. Elle permet aussi de marquer comme inclus un fragment et son complémentaire en même temps.

Par contre les arcs doivent indiquer, pour le nœud source comme pour le nœud destination, quel fragment parmi les deux du nœud il concerne. Ainsi, un arc tel qu'il est stocké dans le tas est un objet qui contient trois attributs entiers correspondant à son nœud source, son nœud destination et son poids, et deux attributs booléens indiquant l'un pour la source l'autre pour la destination quel fragment parmi les deux du nœud est concerné.

Notons que lors de la construction du graphe, on ne va pas reconstruire réellement l'alignement correspondant à chaque arc, mais seulement regarder le score lié à cet alignement pour pouvoir instancier l'arc. La correspondance entre un arc et l'alignement semi-global correspondant est effectuée juste avant l'étape de propagation d'étape. Ceci permet d'éviter de devoir stocker deux fragments (les versions alignées des fragments) pour chaque arc. 
Comme désavantage, il faut recalculer les matrices d'alignement semi-global pour chaque arc du chemin hamiltonien, au moment de la propagation des gaps. Mais le nombre d'arcs concernés par la propagation des gaps est tellement petit par rapport au nombre total d'arcs que ce désavantage est négligeable.

\subsection{Chemin hamiltonien}

L'algorithme utilisé pour construire le chemin hamiltonien est algorithme glouton, très proche de l'algorithme de Kruskal - qui sert à la construction de l'arbre couvrant de poids minimal d'un graphe. L'algorithme parcourt les chemins classés par ordre de poids décroissant.
Pendant ce parcours des arcs, un arc $f \to g$ est accepté dans le chemin hamiltonien si :

\begin{itemize}
\item pour tous les arcs $i \to j$ déjà choisis dans le chemin, $i \neq f'$ et $j \neq f'$  ($f'$ étant le complémentaire-inversé de $f$)
\item pour tous les arcs $i \to j$ déjà choisis dans le chemin, $i \neq g'$ et $j \neq g'$  ($g'$ étant le complémentaire-inversé de $g$)
\item ajouter $f \to g$ au chemin ne va pas créer de cycle
\item pour tous les arcs $i \to j$ déjà choisis dans le chemin, $i \neq f$ et $j \neq g$.
\end{itemize}

L'algorithme s'arrête quand tous les sommets du graphe sont traversés par le chemin créé.

Dans notre cas, la structure qui stocke les arcs est un tas trié de sorte à ce que sa racine soit l'arc de poids le plus important. Pour construire le chemin hamiltonien, nous allons retirer successivement sa racine à ce tas pour parcourir les arcs dans l'ordre décroissant. Cette structure permet l'accès à un arc en $O(log(n))$, et donc si tous les arcs doivent être parcourus,  on aura une complexité dans le pire cas en $O(n*log(n))$ pour le parcours (qui correspond bien à la complexité du tri par tas).

Pour gérer la détection de cycle, on utilise une structure nommée \textit{Union-find}\footnote{https://en.wikipedia.org/wiki/Disjoint-set\_data\_structure}, qui est la structure classiquement utilisée pour le même rôle dans l'algorithme de Kruskal. Cette structure fournit les opérations \textit{find} et \textit{union}. Le \textit{find} permet d'identifier l'ensemble auquel appartient un élément et le \textit{union} de fusionner deux ensembles. Ces deux opérations présentent une complexité avantageuse. Pour détecter les cycles, on assigne donc au départ à chaque nœud son propre ensemble. Lorsqu'un arc $f \to g$ est choisi, l'ensemble contenant $f$ et celui contenant $g$ sont unis. Ainsi un ensemble dans le \textit{Union-find} représente un ensemble de fragments reliés par des arcs ayant été choisis dans le chemin. Pour prévenir la formation d'un cycle lors de l'ajout d'un $f \to g$, on vérifie que l'ensemble qui contient un $f$ soit différent de celui qui contient $g$.

Pour marquer les fragments déjà utilisés dans le chemin, on utilise deux tableaux \textit{in} et \textit{out}. Le tableau \textit{in} marque les nœuds qui sont les destinations d'arcs du chemin, et le tableau \textit{out} marque les nœuds qui sont sources d'arcs du chemin. Comme un nœud contient deux fragments, on utilise deux valeurs différentes dans les tableaux \textit{in} et \textit{out} selon que le fragment soit le complémentaire-inversé ou non. Par exemple, si l'arc $f \to g'$ est ajouté au chemin, \textit{in}$[g]$ sera mis à 2 et \textit{out}$[f]$ sera mis à 1 (les cases du tableaux sont initialisées à 0).

Pendant le parcours, tous les arcs dont la source ou la destination est un fragment marqué comme inclus sont ignorés.

L'utilisation du multithreading à l'étape précédente rend non-déterministe l'ordre dans lequel les arcs sont ajoutés au tas. Pour cette raison, si l'arc utilise comme seul critère de tri le poids, il est impossible de savoir parmi deux arcs de même poids lequel sera considéré avant l'autre pendant le parcours du graphe. L'algorithme étant glouton, ce non-déterminisme a des conséquences importantes sur l'ensemble d'arcs choisi et donc sur le résultat final. Pour éviter cette situation, des critères ont été ajoutés pour trier les arcs. Le poids reste le premier critère, et les critères suivants n'interviennent qu'en cas d'égalité des poids. Ces critères sont arbitraires et ont pour seule utilité de garantir un résultat déterministe. Le premier de ces critères est l'indice du nœud source et le second (qui n'intervient qu'en cas d'égalité des deux précédents) est l'indice du nœud destination.


\subsection{Propagation des gaps}

L'étape de propagation des gaps consiste à aligner les différents fragments selon le chemin hamiltonien construit à l'étape précédente. Cette étape consiste donc à transformer l'ensemble d'arcs du chemin en un ensemble de fragments alignés. 

Pour réaliser cet alignement de nombreux fragments, il faut, pour chaque paire d'arcs consécutifs du chemin hamiltonien (i.e., chaque paire d'arcs $f \to g$ et $g \to h$), aligner le $g$ obtenu dans l'alignement $f \to g$ avec le $g$ obtenu dans l'alignement $g \to h$. A chaque fois qu'un gap est inséré dans le $g$ de l'arc $f \to g$, il faut l'insérer dans tous les fragments précédant $g$ dans le chemin. Similairement, lorsqu'un gap est ajouté dans le $g$ de l'arc $g \to h$, il faut le propager dans tous les fragments suivants $g$ dans le chemin hamiltonien. 

Pour illustrer, prenons les fragments $f =$ acaatgatc, $g =$ caagatcagga et $h =$ agagtcaggacc et considérons le chemin hamiltonien $f \to g \to h$.

Voici l'alignement correspond à l'arc $f \to g$ :  
\begin{boxedverbatim}
acaatgatc----
-caa-gatcagga
\end{boxedverbatim}.

Et celui correspondant à l'alignement $g \to f$ :  
\begin{boxedverbatim}
caaga-tcagga--
--agagtcaggacc
\end{boxedverbatim}.

L'alignement $f \to g \to h$ obtenu après la propagation des gaps est celui-ci : 

\begin{boxedverbatim}
acaatga-tc------
-caa-ga-tcagga--
---a-gagtcaggacc
\end{boxedverbatim}.

Le but est donc de faire correspondre le $g$ du premier arc avec le $g$ du second en ajoutant les gaps nécessaires, tout en propageant ces gaps vers le haut ou vers le bas. On peut par exemple voir un gap ajouté à la 5\up{e} ligne et propagé vers le haut, ou encore un gap ajouté à la $8^e$ ligne, propagé vers le bas.

En résumé, la propagation des gaps consiste à parcourir successivement les arcs du chemin, et pour chaque couple d'arcs successifs
$f \to g$ et $g \to h$, parcourir les deux $g$ obtenus de la gauche vers la droite en ajoutant des gaps en haut ou en bas dans le but d'aligner complètement ces deux $g$. Tous les gaps ajoutés sont propagés respectivement vers le haut ou le bas. Une fois que les $g$ sont complètement alignés, on passe à l'arc suivant.
\\~\\


Ici vont être abordées les spécificités propres à notre implémentation. La première question qui s'est posée au moment d'implémenter la propagation des gaps était relative à la structure de données utilisée pour stocker l'ensemble d'arcs utilisés par le chemin hamiltonien. 
L'algorithme glouton utilisé pour construire le chemin nous donne les arcs par ordre de poids décroissants, et donc pas dans l'ordre qui correspondrait à un parcours de ce chemin. Pour avoir les arcs dans cet ordre et ainsi pouvoir aligner deux à deux les arcs consécutifs, il faut trier les arcs du chemin.

Pour effectuer ce tri, trois solutions ont été envisagées :
\begin{itemize}
\item utiliser un arbre de recherche équilibré (\textit{TreeSet}\footnote{https://docs.oracle.com/javase/7/docs/api/java/util/TreeSet.html} de java). Cette structure garantit l'insertion et la recherche d'un élément en en $O(log(n))$. L'idée est de stocker les arcs dans l'arbre en les triant en fonction de leur source. Ainsi, une fois qu'on a trouvé le premier arc du chemin, on peut trouver les $n-1$ suivants en $O(n(log(n))$.

\item utiliser une table de hachage (\textit{HashMap}\footnote{https://docs.oracle.com/javase/8/docs/api/java/util/HashMap.html} de java) permettant de retrouver un arc en fonction de sa source. En utilisant une table de hachage ayant pour clés les sources des arcs et comme valeurs les arc, retrouver un arc en fonction de sa source se fait avec une complexité moyenne en $O(1)$ (bien qu'aucune garantie ne soit donnée d'avoir cette complexité pour un pire cas). Avec cette solution, trouver les $n-1$ derniers arcs du chemin se fait en moyenne en $O(n(log(n))$.

\item utiliser une table de hachage permettant de retrouver un arc en fonction de sa destination ou de sa source. Ici, l'idée est d'ajouter deux entrées dans la table pour chaque arc : une entrée associant cet arc à sa source et l'autre à sa destination. Les clés sont donc des objets constitués soit de la source et d'un booléen valant $true$, soit de la destination et d'un booléen valant $false$. Les booléens permettent de distinguer les clés qui correspondent aux sources des clés qui correspondent aux destinations. Les valeurs sont les arcs. Cette solution permet de ne pas avoir à trouver le premier arc du chemin, puisqu'il est possible, partant d'un arc quelconque, de retrouver aussi bien l'arc suivant que l'arc précédent en $O(1)$ (en moyenne). Ainsi, il suffit de choisir un arc quelconque et trouver les $n-1$ autres arcs se fait en moyenne en $O(n(log(n))$.
\end{itemize}

C'est finalement la deuxième solution qui a été implémentée. D'après sa complexité, la première a été rapidement écartée (le pire cas des tables de hachage étant évitable grâce à un bon choix de la fonction de hachage et de la taille de la table). 
La troisième solution a comme avantage sur la deuxième de permettre de commencer à partir d'un fragment quelconque. Par contre, elle nécessite une table occupant deux fois plus d'espace et elle rend plus complexe l'implémentation. La difficulté supplémentaire d'implémentation est liée au fait que si les arcs sont traités en partant du premier, propager les gaps vers le bas revient à insérer des gaps dans un seul fragment, alors que traiter les arcs en suivant le chemin à l'envers implique de pouvoir propager les gaps vers le bas sur plusieurs fragments.

En itérant sur la liste des clés de la table qui contient le chemin, et en comparant chacune de ces clés avec l'entrée correspondante dans le tableau \textit{in} utilisé pour l'algorithme glouton, on peut trouver le premier arc du chemin en $O(n+m)$ (où $n$ est le nombre d'arcs et $m$ est la capacité de la \textit{HashMap}\footnote{\textcolor{red}{D'après la documentation java}}). En effet, la source du premier chemin est le seul nœud qui n'est pas une destination d'un arc du chemin, et donc le seul pour lequel l'entrée du tableau \textit{in} vaudra $0$. \textcolor{red}{dire que la capacité de la hashmap vaut n/0.75 car 0.75 est le facteur de charge par défaut, et que nous l'avons laissé comme ça}.

Au vu du peu de temps nécessaire pour trouver le premier fragment, et tenant compte des autres avantages et défauts des deuxièmes et troisièmes solutions, nous avons choisi d'implémenter la deuxième.
\\~\\ 
 
Après avoir choisi le moyen de trier nos arcs, s'est posé le problème du choix de la structure utilisée pour stocker les fragments en train d'être alignés. Pour cela, il faut une structure permettant facilement l'insertion d'un caractère à un indice donné. Les \textit{LinkedList}\footnote{https://docs.oracle.com/javase/7/docs/api/java/util/LinkedList.html} de java, qui sont des listes doublement chaînées, ont été choisies. Dans une première implémentation, naïve, une \textit{LinkedList} était utilisée pour stocker l'ensemble des caractères des fragments alignés. 
Bien que peu coûteuse en temps sur les petites instances, cette implémentation est qu'elle s'est avérée extrêmement peu efficace pour la grande collection - plus d'une demi-heure nécessaire pour propager les gaps. 

Afin d'améliorer cette première implémentation, plusieurs modifications ont été apportées.

La première d'entre elles, et la plus importante en termes d'impact sur la complexité, est la modification de la structure utilisée pour stocker les fragments en cours d'alignement (classe \textit{FragmentBuilder}). Plutôt que d'utiliser une liste chaînée pour stocker l'ensemble des caractères de chaque \textit{FragmentBuilder}, nous avons décidé d'utiliser une structure constituée de deux entiers et d'une liste chaînée. Pour illustrer les motivations ayant amené ce changement, voyons cet exemple nous montrant un alignement de 7 fragments.

\begin{verbatim}
tccgaagtctgct----------------------------------
---------tgctgctggag---------------------------
----------actactag-gcc-------------------------
----------------ag-gtcaactgatc-----------------
---------------------caactg-ccaaaaa------------
----------------------aac---ccaaaaagggg--------
-----------------------------------gggggggtcggt
\end{verbatim}

On constate que la majorité des caractères que chacun de ces fragments sont des gaps situés à la fin ou au début du mot. Plus l'exemple est grand, plus la proportion de gaps est importante. Pour le plus grand exemple, tous les mots alignés comptent environ $100000$ caractères, alors que chacun d'entre eux provient d'un fragment qui compte en $500$ et $700$ caractères. Sur les $100000$ caractères, on a donc énormément de gaps situés avant ou après une zone d'environ $700$ caractères. L'idée est d'éviter d'utiliser $n$ nœuds de \textit{LinkedList} (un Byte dans chaque nœud) pour stocker tous ces gaps.
Chaque \textit{FragmentBuilder} a donc deux entiers \textit{startGaps} et \textit{endGaps} qui stockent respectivement le nombre de gaps situés au début et à la fin du mot. Pour stocker les caractères situés entre ces gaps, une \textit{LinkedList} est toujours utilisée. Ce changement permet d'économiser de la mémoire en utilisant seulement deux entiers là où étaient nécessaires jusqu'à plus de $99000$ nœuds de \textit{LinkedList} auparavant. 

Une autre amélioration est l'utilisation de \textit{ListIterator} pour itérer sur les listes chaînées en $O(n)$. En effet atteindre le $i^e$ d'une liste chaînée demande de parcourir les $i-1$ éléments précédents (éventuellement moins dans le cas d'une liste non chaînée, mais la complexité reste inchangée). Atteindre le $i^e$ élément de la liste pour $i$ allant de 1 à $n$ donne donc une complexité du parcours en $0(n^2)$.

Pour aligner les deux $g_i$ des arcs $f \to g_1$ et $g_2 \to h$, on ajoute directement \textit{startGaps}$(g_1)$ gaps au début de $g_2$ et ensuite comparer le restant du $g1$ et $g2$ à partir de cet indice. Cette opération est possible parce que, par définition de $g2 \to h$, $g2$ ne peut pas commencer par des gaps. Ceci évite un grand nombre de comparaisons inutiles.

Dernière amélioration notable, on évite d'ajouter un par un les gaps de fin lorsqu'on propage vers le haut. En effet, ces gaps ne servent qu'à faire en sorte que les mots aient la même longueur. Pour avoir malgré tout un tableau de fragments alignés à la fin, on ajoute autant de gaps que nécessaire à la fin des mots pour qu'ils aient tous la même longueur que le plus long d'entre eux -  qui est le dernier du tableau. Ainsi, plutôt que d'incrémenter \textit{endGaps} $x$ fois, on incrémenter \textit{endGaps} de $x$ une seule fois.

Avec toutes ces modifications, l'étape de propagation se fait en $O(n*m)$, où $n$ est la taille du chemin hamiltonien et $m$ la taille maximale d'un fragment. En effet, on traite les $n$ fragments du chemin, et pour chacun d'eux on parcourt (en temps linéaire) uniquement la zone située entre les gaps du début et les gaps de fin. Dans la version précédente, ajouter un nouveau mot demandait d'ajouter un par un les $n$ gaps du début, et de propager sur tous les fragments déjà traités les gaps de fin. Si pour les petites instances le gain de temps n'a pas été déterminant - bien que non négligeable -, les changements ont permis de réduire le temps de propagation des gaps de plus d'une demi-heure à moins d'une demi-seconde.


\subsection{Vote de consensus}

Dernière de l'assemblage, le vote de consensus est l'étape où est construit le fragment qui sera donné en sortie. Il s'agit de retourner un fragment tel que son $i^e$ caractère est le caractère le plus présent dans la $i^e$ colonne du tableau construit pendant lors de la propagation des gaps.

Pour cette étape, notre implémentation consiste simplement à compter, pour chaque colonne, le nombre d’occurrences de chaque caractère (en ignorant les gaps) et à placer ensuite ce caractère au bon indice dans la séquence résultat.
Pour éviter d'utiliser inutilement de la mémoire en instanciant des tableaux de caractères, nous avons choisi de conserver des \textit{FragmentBuilder} pour cette étape. Pour conserver l'accès en temps constant à un élément, des \textit{Iterator} sont utilisés pour parcourir les \textit{LinkedList} des différents \textit{FragmentBuilder}.

Une petite optimisation est issue du fait que pour deux fragments alignés $f$ et $g$, avoir $f$ situé plus haut que $g$ dans le tableau \textit{FragmentBuilder} implique que \textit{startGaps}$(f) < $  \textit{startGaps}$(g)$ et \textit{endGaps}$(f) > $  \textit{endGaps}$(g)$. La première implication signifie que pour une colonne donnée, on peut s'arrêter de descendre dès qu'on arrive dans les gaps du début d'un mot, puisque chacun des mots situés plus bas aura un gap dans à cet indice. Similairement, la seconde implication permet de commencer à la ligne $j$ pour une colonne $i$ si l'indice $i-1$ se trouve dans la zone des gaps de fin de la $j^e$ ligne. 

En cas d'égalité entre deux caractères, le caractère résultant est choisi selon l'ordre arbitraire suivant (ici classés par ordre de préférence décroissante) : a, t, c, g.

\section{Qualité du projet}

\subsection{Efficacité en temps en mémoire}
Pour évaluer la qualité de nos résultats, les séquences cibles liées à nos collections de test 

Les tests unitaires, bien que non-exhaustifs, permettent quasiment d'avoir la certitude du bon fonctionnement de nos méthodes.

\subsection{\'Evaluation des résultats}

Pour am


\section{Conclusion}

Bien que les dotmatchers


Une piste d'amélioration consisterait à améliorer l'algorithme glouton de construction du chemin hamiltonien. En effet, actuellement un temps important est utilisé pour construire un graphe complet, pour que finalement les arcs soient choisis 

(+++ j'ai pas fait le blabla dans le SGA juste la table (pour l'instant je m'en occupe si tu veux)  - est ce que scinder l'intro est une bonne idée ? - est ce qu'il fut rajouter du contenu dans l'intro ? -  la répartition du travail est brève je sais aps si ça leur suffira -  j'ai refais le vote de consensus -  qualité du projet incomplet selon moi - pas d'inspi pour la conclusion +++)

utilisation de structure donné nouvelle

progression nette7

cependant - amélioration du résultat (lacune en bio)


\end{document}